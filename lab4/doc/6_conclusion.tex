\Conclusion
    В ходе выполнения лабораторной работы были достигнуты следующие задачи:
    \begin{enumerate}
        \item изучен алгоритм обратной трассировки лучей;
        \item реализован алгоритм обратной трассировки лучей;
        \item проведены замеры процессорного времени работы от разного числа параллельных потоков;
        \item проведено сравнение параллельных реализаций алгоритма обратной трассировки лучей со стандартной реализацией.
    \end{enumerate}

В ходе сравнения процессорного времени работы было установлено:
	\begin{enumerate}
	\item параллельная реализация алгоритма обратной трассировки лучей в случае увеличения числа потоков в 2 раза, начиная с 4 замедляет скорость своей работы;
	\item максимальный прирост скорости параллельной реализации алгоритма обратной трассировки лучей достигается при количестве потоков равных 4 и равен  $ \approx 1.7 $ раза.


        \end{enumerate}

Эксперементальным путем было выяснено, что увеличение количества потоков в 2 раза не всегда увеличивает производительность программы.

    


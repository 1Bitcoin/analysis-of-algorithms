\Introduction    
Трассировка лучей (англ. Ray tracing; рейтрейсинг) — один из методов геометрической оптики — исследование оптических систем путём отслеживания взаимодействия отдельных лучей с поверхностями. В узком смысле — технология построения изображения трёхмерных моделей в компьютерных программах, при которых отслеживается обратная траектория распространения луча (от экрана к источнику).

    Целью данной лабораторной работы является изучение и реализация параллельных вычислений
    для обратного трассировщика лучей.

    Задачи данной лабораторной работы:
    \begin{enumerate}
        \item изучить алгоритм обратной трассировки лучей;
        \item реализовать алгоритм обратной трассировки лучей;
        \item провести замеры процессорного времени работы от разного числа параллельных потоков;
        \item провести сравнение параллельных реализаций алгоритма обратной трассировки лучей со стандартной реализацией.
    \end{enumerate}

\newpage
\chapter{Экспериментальный раздел}
\label{cha:research}
    В данном разделе будут выполнены эксперименты для проведения 
    сравнительного анализа алгоритмов обратной трассировки лучей по затрачиваемому процессорному 
    времени в зависимости от числа потоков.

    \section{Сравнительный анализ на основе замеров времени работы алгоритмов}
	 В рамках данного проекта был произведен замер времени работы алгоритма обратной трассировки лучей при количестве потоков равном 1, 2, 4, 8, 16, 32, 64.
	Результаты данных измерений приведены на графике \ref{graph:1}.

        Тестирование проводилось на ноутбуке с процессором
        Intel(R) Core(TM) i3-8130U CPU \cite{processor-i3-8130u}, 4 логических процессора,
        под управлением Windows 10 с 8 Гб оперативной памяти.

    \begin{figure}[h!]
        \centering
        \begin{tikzpicture}
            \begin{axis}[
                legend pos = north west,
                grid = major,
                xlabel = Число потоков,
                ylabel = {Время, миллисекунды},
                height = 0.5\paperheight, 
                width = 0.75\paperwidth
            ]
            
            \addplot table[x=n,y=time] {data/workTime.dat};
            \end{axis}
        \end{tikzpicture}
        \caption{График зависимости времени работы алгоритма обратной трассировки лучей от числа потоков} 
        \label{graph:1}
    \end{figure}

        В ходе экспериментов по замеру времени работы было установлено:
	\begin{enumerate}
	\item параллельная реализация алгоритма обратной трассировки лучей в случае увеличения числа потоков в 2 раза, начиная с 8 замедляет скорость своей работы;
	\item максимальный прирост скорости параллельной реализации алгоритма обратной трассировки лучей достигается при количестве потоков равных 4 и равен  $ \approx 1.7 $ раза.

        \end{enumerate}



   
\newpage
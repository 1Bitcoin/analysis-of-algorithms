\chapter{ Технологический раздел}
\label{cha:technological}

    В данном разделе будут выбраны средства реализации ПО и представлен листинг кода. 

    \section{Средства реализации}
        В данной работе используется язык программирования C\# \cite{Charp}, так как
        он позволяет написать программу в относительно малый срок.
        В качестве среды разработки использовалась Visual Studio 2017 \cite{visual-studio}. Для замера процессорного времени был использован класс Stopwatch \cite{stopwatch}.

	Многопоточное программирование было реализовано с помощью пространства имен System.Threading \cite{thread}.


        \begin{lstlisting}[label=lst:rtx_parall, caption=Реализация алгоритма обратной трассировки лучей с параллельными вычислениями]
            int n = 4; // Количество потоков

            int step = data.Width / n;

            Thread[] t = new Thread[n];
       
            int x1 = 0;
            int x2 = step;

            int y1 = 0;
            int y2 = data.Height;
         
            for (int i = 0; i < n; i++)
            {
                AllParameters p = new AllParameters(x1, y1, x2, y2);

                t[i] = new Thread(Process);
                t[i].Start(p);

                x1 = x2;
                x2 += step; 
            }

            foreach (Thread thread in t)
            {
                thread.Join();
            }

            void Process(object obj)
            {
                AllParameters p = (AllParameters)obj;

                int x = p.x;
                int y = p.y;

                int endx = p.endx;
                int endy = p.endy;

                for (int i = x; i < endx; i++)
                {
                    for (int j = y; j < endy; j++)
                    {
                        int[] work = { i - data.Height / 2, j - data.Width / 2 };

                        double[] direction = RayTracing.CanvasToViewport(data.Width, data.Height, work);
                        direction = MyMath.MultiplyMV(cameraRotationOY, direction);

                        double[] color = RayTracing.TraceRay(recursionDepth, lights, objects, cameraPosition, direction, 1, Double.PositiveInfinity, 0);

                        var offset = ((j * data.Width) + i) * depth;

                        buffer[offset + 0] = (byte)color[2];
                        buffer[offset + 1] = (byte)color[1];
                        buffer[offset + 2] = (byte)color[0];
                    }
                }

            }
        \end{lstlisting}

    

\newpage
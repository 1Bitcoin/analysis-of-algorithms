\chapter{ Аналитический раздел}
\label{cha:analytical}
    В данном разделе будут рассмотрены основные теоритические понятия конвейерной обработки и параллельных вычислений.

    \section{Основные понятия}
Конвейер — способ организации вычислений, используемый в современных процессорах и контроллерах с целью повышения их производительности (увеличения числа инструкций, выполняемых в единицу времени).

Один из самых простых и наиболее распространенных способов повышения быстродействия процессоров — конвейеризация процесса вычислений.

Конвейеризация – это техника, в результате которой задача или  команда разбивается  на некоторое число подзадач, которые  выполняются последовательно.
Каждая  подкоманда   выполняется на своем логическом  устройстве.
Все логические устройства (ступени) соединяются последовательно таким образом, что выход i-ой  ступени  связан   с   входом   (i+1)-ой   ступени,  все ступени  работают  одновременно.
Множество  ступеней называется конвейером. Выигрыш во времени достигается при выполнении  нескольких задач  за  счет параллельной   работы   ступеней,  вовлекая  на  каждом такте новую задачу или команду.

Идея заключается в параллельном выполнении нескольких инструкций процессора. Сложные инструкции процессора представляются в виде последовательности более простых стадий. Вместо выполнения инструкций последовательно (ожидания завершения конца одной инструкции и перехода к следующей), следующая инструкция может выполняться через несколько стадий выполнения первой инструкции. Это позволяет управляющим цепям процессора получать инструкции со скоростью самой медленной стадии обработки, однако при этом намного быстрее, чем при последовательном выполнении каждой инструкции от начала до конца.

    \section{Оценка производительности конвейера}
Пусть задана операция, выполнение которой разбито на n последовательных этапов. При последовательном их выполнении операция выполняется за время

\begin{equation}\label{form:way}
 \tau _{e}={\sum\limits_{i=1}^n \tau _{i}}
 \end{equation}
 \begin{align*}
    \text{где} \\
    n &- \text{количество последовательных этапов;} \\
   \tau _{i} &- \text{время выполнения i-го этапа;}
\end{align*}

Быстродействие одного процессора, выполняющего только эту операцию, составит

\begin{equation}\label{form:way}
 S_{e}={\frac{1}{\tau _{e}}}={\frac{1}{\sum\limits_{i=1}^n \tau _{i}}}
 \end{equation}
 \begin{align*}
    \text{где} \\
    \tau _{e} &- \text{время выполнения одной операции;} \\
    n &- \text{количество последовательных этапов;} \\
   \tau _{i} &- \text{время выполнения i-го этапа;}
\end{align*}

Выберем время такта — величину $t _{T} = max{\sum\limits_{i=1}^n(\tau_{i})}$ и потребуем при разбиении на этапы, чтобы для любого i = 1, ... , n выполнялось условие $(\tau_{i} + \tau_{i+1}) mod n = \tau_{T}$. То есть чтобы никакие два последовательных этапа (включая конец и новое начало операции) не могли быть выполнены за время одного такта.

Максимальное быстродействие процессора при полной загрузке конвейера составляет

\begin{equation}\label{form:way}
 S_{max}={\frac{1}{\tau _{T}}}
 \end{equation}
 \begin{align*}
    \text{где} \\
    \tau _{T} &- \text{выбранное нами время такта;}
\end{align*}

Число n — количество уровней конвейера, или глубина перекрытия, так как каждый такт на конвейере параллельно выполняются n операций. Чем больше число уровней (станций), тем больший выигрыш в быстродействии может быть получен.

Известна оценка
\begin{equation}\label{form:way}
{\frac{n}{n/2} \leq {\frac{S_{max}}{S_{e}}} \leq n}
 \end{equation}
 \begin{align*}
    \text{где} \\
    S_{max} &- \text{максимальное быстродействие процессора  при полной загрузке конвейера;} \\
    S_{e} &- \text{стандартное быстродействие процессора;} \\
   n &- \text{количество этапов.}
\end{align*}

то есть выигрыш в быстродействии получается от n/2  до n раз [2].


Реальный выигрыш в быстродействии оказывается всегда меньше, чем указанный выше, поскольку:

\begin{enumerate}
\item[1)] некоторые операции, например, над целыми, могут выполняться за меньшее количество этапов, чем другие арифметические операции. Тогда отдельные станции конвейера будут простаивать;
\item[2)] при выполнении некоторых операций на определённых этапах могут требоваться результаты более поздних, ещё не выполненных этапов предыдущих операций. Приходится приостанавливать конвейер;
\item[3)] поток команд(первая ступень) порождает недостаточное количество операций для полной загрузки конвейера.
\end{enumerate}
        
    \section{Вывод}
        В данном разделе были рассмотрены основы конвейерной обработки и технологии параллельного программирования.

\newpage
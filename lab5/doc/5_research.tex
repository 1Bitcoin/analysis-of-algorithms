\chapter{Экспериментальный раздел}
\label{cha:research}
    В данном разделе будут проведены эксперименты для проведения 
    сравнительного анализа на основе замеров времени работы алгоритмов.

    \section{Сравнительный анализ на основе замеров времени работы алгоритмов}
        В рамках данной работы был проведен эксперимент по вычислению времени работы системы в линейной и конвейерной реализациях (график \ref{graph:test:models}).



        Тестирование проводилось на ноутбуке с процессором
        Intel(R) Core(TM) i3-8130U CPU 2.20 GHz \cite{processor-i5-7200u}
        под управлением Windows 10 с 8 Гб оперативной памяти.

        В ходе эксперимента по замеру времени работы в линейной и конвейерной реализациях было установлено,
        что конвейерная модель обрабатывает элементы в $ \approx 2.6 $ раза
        быстрее, чем линейная. Это объясняется тем,
        что одновременно обрабатываются разные элементы на разных этапах.
        



    \begin{figure}[h!]
        \centering
        \begin{tikzpicture}
            \begin{axis}[
                legend pos = north west,
                grid = major,
                xlabel = Количество элементов,
                ylabel = {Время, мс},
                height = 0.5\paperheight, 
                width = 0.75\paperwidth
            ]
            
            \addplot table[x=n, y=linear] {data/time-models.dat};
            \addplot table[x=n, y=conveyor] {data/time-models.dat};
            \legend{
                Линейная,
                Конвейерная
            };
            \end{axis}
        \end{tikzpicture}
        \caption{График зависимости времени работы от модели системы} 
        \label{graph:test:models}
    \end{figure}



\newpage